%!TEX root = sib_prog_eval_two_columns.tex


We assess the performance of four programs for inference of full sibling groups from
genetic data: \colony{} \citep{wang04,Wang2012pairwise}, \prt{} 
\citep{almudevar99}, \familyfinder{} \citep{beyer03}, and \kinalyzer{} 
\citep{bergerwolf07, sheik08}. We used each program to infer full siblings
in over ten thousand data sets of simulated microsatellite  
genotypes from individuals falling into 10 distinct relationship configurations
that differed in the number of individuals related as full- and half-siblings. 
The accuracy of
each method was assessed by calculating the partition distance between the true partition
of the sample into full sibling groups and the inferred partition. Across almost all
scenarios in which full siblings were present, \colony{} (version 2.0.5.7) performed as well or better than
any of the other programs; however we note that it can exhibit a tendency to report 
small, erroneous, ``high-confidence'' full sibling groups from data sets in which there
are no full siblings. In particular, caution is warranted in identifying sibling
{\em pairs} using \colony{}.  \prt{} outperformed \colony{} version 2.0 across a range of 
locus and allele numbers in data sets that lacked any real full siblings.  However, this advantage
was largely erased following the improvements made in \colony{} version 2.0.5.7. 
When data were very informative (many loci with many alleles), then
\familyfinder{}, which was orders of magnitude faster than any of the other programs,
was often as accurate as the most accurate program, so long as no
half-siblings appeared in the data.  When half-siblings
were present, \familyfinder{} tended to erroneously identify
them as full-siblings, and the problem worsened with more markers and alleles.
There were no scenarios in which \kinalyzer{} consistently outperformed the other methods.


\documentclass[twoside,10pt,twocolumn]{article}
% ------
% Fonts and typesetting settings
\usepackage[sc]{mathpazo}
%\usepackage[T1]{fontenc}
\linespread{1.05} % Palatino needs more space between lines

\usepackage{microtype}
% ------
% Page layout
\usepackage[labelfont=bf,labelformat=simple,labelsep=quad]{caption}

\usepackage{graphicx}
\usepackage{amssymb}
\usepackage{epstopdf}
\usepackage{amsfonts}
\usepackage{natbib}
\usepackage{subfigure}
\usepackage{pdfsync}
\usepackage{xspace}
\usepackage{mathrsfs}
\usepackage{fancyhdr}
\usepackage{cuted}
\usepackage{flushend}
\usepackage{url}
\usepackage{color}
%\usepackage{balance}
\usepackage{xr}


%% some handy things for making bold math
\def\bm#1{\mathpalette\bmstyle{#1}}
\def\bmstyle#1#2{\mbox{\boldmath$#1#2$}}
\newcommand{\thh}{^\mathrm{th}}


%% Some pretty etc.'s, etc...
\newcommand{\cf}{{\em cf.}\xspace }
\newcommand{\eg}{{\em e.g.},\xspace }
\newcommand{\ie}{{\em i.e.},\xspace }
\newcommand{\etal}{{\em et al.}\ }
\newcommand{\etc}{{\em etc.}\@\xspace}


%% the page dimensions from TeXShop's default---very nice
\textwidth = 6.5 in
\textheight = 9 in
\oddsidemargin = -0.01 in
\evensidemargin = -0.01 in
\topmargin = -0.7 in
\headheight = 0.25 in
\headsep = 0.25 in
\parskip = 0.0in
\parindent = 0.25in

\setlength{\columnsep}{.4in}



%% change section heading styles
\makeatletter
\renewcommand\section{\@startsection {section}{1}{\z@}%
                                   {-3.5ex \@plus -1ex \@minus -.2ex}%
                                   {1.5ex \@plus 0ex}%
                                   {\normalfont\normalsize\bfseries}}
\renewcommand\subsection{\@startsection{subsection}{2}{\z@}%
                                     {-3.25ex\@plus -1ex \@minus -.2ex}%
                                     {1.5ex \@plus .2ex}%
                                     {\normalfont\normalsize\itshape}}
\makeatother


%\fancyhead{} % clear all header fields
%\chead[]{\vspace*{-.15in}\includegraphics[width=\textwidth]{banner.pdf}}
%\cfoot[]{}
%\fancyhead[LE]{{\bf \thepage}~~{\sl SUPPORTING INFORMATION FOR:} ANDERSON \& ALMUDEVAR}
%\fancyfoot[RE,LO]{{\footnotesize This supporting information is a US Government work and is in the public domain in the USA.}}
%\renewcommand{\headrulewidth}{0pt}


%% Commands for some mathematical notation:
\newcommand{\dates}{\mathscr{D}}
\newcommand{\observ}{\mathscr{O}}
\newcommand{\latent}{\mathscr{L}}
\newcommand{\bz}{\bm{z}}
\newcommand{\btheta}{\bm{\theta}}
\newcommand{\bS}{\bm{S}}

%% AA \newcommands 

\newcommand{\bcaa}{\begin{color}{green}}
\newcommand{\ecaa}{\end{color}}
\newcommand{\calp}{{\cal P}}

%\externaldocument{../herring_revision1}

\begin{document}


%\pagestyle{fancyplain}



   \begin{strip}
   \vspace*{-.1in}
   \mbox{}\\
   \mbox{}\\
        {\LARGE\bf An Assessment of Sibship Reconstruction Programs with Simulated Data \par}
    \mbox{}\\
    \uppercase{ Eric C. Anderson$^*$ and Anthony Almudevar$^\dagger$} \\
       \mbox{}\\
    $^*${\em Fisheries Ecology Division, Southwest Fisheries Science Center, Santa Cruz, CA, 
    USA, ~$^\dagger$Department of Biostatistics and Computational Biology,
		University of Rochester Medical School, Rochester, NY}\\
    \mbox{}\\
    These appendices authored by Anthony Almudevar.\\
    Correspondence regarding these appendices: Anthony\_Almudevar@URMC.Rochester.edu
    \begin{abstract}
      Boing Boing Boing
    \end{abstract}
\vspace*{.4in}
   \end{strip}


\section*{Appendix 1 - New features of PRT V2.2} 

A number of features have been added to PRT, constituting realease V2.2, available at  
\url{www.urmc.rochester.edu/biostat/people/faculty/almudevar.cfm}  (the immediately preceding release V2.0 
is described in \citet{alm&and11}).

\subsection*{Genotyping errors}

V2.0 introduced \emph{post hoc} genotype error screening.  Following a partition estimate, each individual 
is checked to see if it can form a FSG with each other SG at all loci except exactly one. All such instances 
are reported. V2.2 introduces genotyping error tolerance into the partition estimate itself. An optional 
value $K$ is chosen representing the maximum number of genotype errors permitted per individual ($K=0$ 
forces no error tolerance). If genotype errors are flagged, they are reset to the missing value code. The 
algorithm is then repeated with the modified data. This process may be iterated any number of times.   The 
value $K$ is ideally chosen to conform to the anticipated error rate. 

\subsection*{Memory management}

The method by which PRT allocates memory has been significantly redesigned. The user now allocates a fixed 
block of random access memory in advance. In \citet{almudevar99}, an \emph{Atlantic salmon} data set of 760 
individuals \citep{oreilly98, herbinger99} was used to demonstrate a modification of the MSG algorithm 
designed for large data sets. As reported in \citet{almudevar99} the memory burden on the unmodified 
algorithm was too great, but the modified algorithm was able to complete the problem.  Currently, we have 
found that allocating 750MB is sufficient for the completion of the \emph{Atlantic salmon} data set 
partition without requiring the large sample modification (see also \citealt{alm&and11} for a discussion of 
this issue).  However, the modified algorithm is still retained, as we find it is considerably less time 
consuming, with little apparent loss in accuracy, provided the parameters are suitably chosen (see PRT 
documentation for recommendations on this point).   
  
\subsection*{Batch mode}

While PRT V2.x is designed as a GUI driven application, a batch mode has been implemented. This is invoked 
by reading a script file containing macro commands. Each control of the GUI is assigned a macro command, 
consisting of a control label, with optional parameters as required. When the script file is read, each 
macro command is interpreted in sequence, directly modifying GUI options or activating control buttons as 
appropriate (see PRT documentation for detailed instructions).   

\section*{Appendix 2 - A note on partition distances} 

It is important to consider carefully the selection of the partition distance.  One natural metric is the 
number of incorrect pairwise kinships (which can be further decomposed into false positives and false 
negatives). The problem with this type of metric, at least when used without adjustment, is that it is 
highly sensitive to partition features which are not in error. For example, if a partition estimate is 
correct except for splitting of one individual from a true sibling group of size $m$, then the number of 
errors will be scored as $m-1$, rendering comparisons difficult to interpret. The PD, on the other hand, 
will assign an error of 1 to this estimate, offering the intuitive intepretation that one single individual 
has been incorrectly assigned.      

However, the PD is a relatively coarse metric, particularly when partition differences are accounted for by 
more than a few individuals. An example is offered in \citet{BrownDexter2012}. Suppose a partition is 
correct except that a single true sibling group of size $m$ is split into two groups of size $k$ and $m-k$ 
(taking $k \leq m-k$). Then PD = $k$. Next, suppose the error is compounded by joining the size $k$ group  
to another true sibling group. The number of errors has clearly increased (there are now false positives in 
addition to the original false negatives), but PD remains $k$. 


Despite these concerns, we cannot accept the metric proposed in \citet{BrownDexter2012} as an alternative, 
which we will briefly discuss. Two partitions $\calp = (P_1, \ldots, P_p)$ and $\calp^\prime = (P^\prime_1, 
\ldots, P^\prime_{p^\prime})$ of $\left\{ 1, \ldots, n\right\}$ induce probability distributions $P(i) = |
P_i|/n$ and $P^\prime(i^\prime) = |P^\prime_{i^\prime}|/n$ on sets $\left\{ 1, \ldots, p\right\}$ and $\left
\{ 1, \ldots, p^\prime\right\}$ respectively, where $|P_i|$ denotes the number of elements in set $i$ of 
partition $\calp$. In addition, the joint distribution  $P^*(i,i^\prime) = |P_i \cap P^\prime_{i^\prime}|/n$ 
is defined on  $\left\{ 1, \ldots, p\right\} \times \{ 1, \ldots, p^\prime \}$. The respective 
\emph{entropies} of  $\calp$ and $\calp^\prime$ are
\begin{eqnarray}
H(\calp) &=& - \sum_{i=1}^p P(i) \log P(i) \\
H(\calp^\prime) &=& - \sum_{i^\prime =1}^{p^\prime} P^\prime(i^\prime) \log P^\prime(i^\prime).
\end{eqnarray}
We may also associate an entropy with the joint probability distribution $P^*$:
\begin{displaymath}
H(\calp, \calp^\prime) = - \sum_{i=1}^p \sum_{i^\prime=1}^{p^\prime} P^*(i, i^\prime) \log P^*(i, i^\prime),
\end{displaymath}
where the standard convention in entropy calculations is to interpret $(0\log 0)$ as 0. The entropy of a 
distribution defined on $m$ outcomes ranges from 0 (for a distribution which assigns a probability of 1 to a 
single outcome) to $\log m$ (for a distribution which assigns a probability of $1/m$ to each outcome). The 
\emph{mutual information} of  $\calp$ and $\calp^\prime$ is defined as 
\begin{eqnarray*}
I(\calp, \calp^\prime) & = & \sum_{i=1}^p \sum_{i^\prime=1}^{p^\prime} P^*(i, i^\prime) \log \frac{P^*(i, i^
\prime)}{P(i)P^\prime(i^\prime)} \\
	& = & H(\calp) + H(\calp^\prime) - H(\calp, \calp^\prime).
\end{eqnarray*}
This leads to a related quantity, generally referred to as the \emph{variation of information}
\begin{eqnarray*}
VI(\calp, \calp^\prime) & = & \{ H(\calp) - I(\calp, \calp^\prime) \}  \\
&\mbox{}& ~~ \mbox{}+  \{ H(\calp^\prime) - I(\calp, \calp^\prime) \} \\
& = & H(\calp, \calp^\prime) -  I(\calp, \calp^\prime).
\end{eqnarray*}
It is well known that  $VI(\calp, \calp^\prime)$ is a true metric (it attains a minimum value of 0 if and 
only if $\calp = \calp^\prime$, it is symmetric, and it satisfies the triangle inequality). Also, $VI(\calp, 
\calp^\prime)$ is bounded above by $H(\calp, \calp^\prime)$. In this application, because there are no more 
than $n$ possible outcomes, the joint entropy $H(\calp, \calp^\prime)$ has a sharp upper bound of $\log n$ 
(this bound is achievable if at least one of $p, p^\prime = n$). Then $VI(\calp, \calp^\prime)$ attains the 
upper bound of $\log n$ if $p = n$, $p^\prime = 1$. 




%% A little hack to stop the banner from printing:
%\chead[]{}
%\fancyhead[RO]{{\sl SUPPORTING INFORMATION FOR:} ``AN ASSESSMENT OF SIBSHIP RECONSTRUCTION\ldots''~~{\bf \thepage}}
 

Given these properties, \citet{BrownDexter2012} propose its use as a partition distance, further introducing 
a normalization scheme to force the metric into the unit interval. We note that they propose an extended 
construction for nested half-sibling structures, however, for our discussion it suffices to consider 
partitions only. 


\begin{table}
\caption{Schematic representations of example partitions $\calp_1$, $\calp_2$, $\calp_3$ and $\calp_4$ for 
$n = 6$.}\label{pex.table}
\begin{center}
\begin{tabular}{r|c|c}\hline
$\calp_1$ &AAAAAA&BBBBBB \\
$\calp_2$ &AAACCC&BBBBBB \\
$\calp_3$ &AAAAAB&BBBBBB \\
$\calp_4$ &AAAAAB&ABBBBB 
\end{tabular}
\end{center}
\end{table}



%%%%%%%%%%%%%%%
\begin{table*}
\caption{Values of \emph{variation of information} distance involving sample partitions $\calp_1, \calp_2, \calp_3, \calp_4$ for varying values of $n$ (natural logarithms are used).}\label{vi.table}
\rule{\textwidth}{0.90pt}
\begin{center}
\begin{tabular}{r|ccccccccccc}
\multicolumn{1}{r}{$n=$} &   4  & 6  & 8  & 10  & 12  & 14  & 16  & 18  & 20  & 22  & 24 \\ \hline
$VI(\calp_1, \calp_2)$ & 0.347  &  0.347  & 0.347  &  0.347  & 0.347   & 0.347   & 0.347   & 0.347   & 0.347   & 0.347   & 0.347 \\
$VI(\calp_1, \calp_3)$ & 0.594  & 0.465  & 0.385   & 0.330   & 0.290   & 0.260   & 0.236   & 0.216   & 0.200   & 0.186   & 0.174 \\
$VI(\calp_1, \calp_4)$ & 1.125  & 0.901  & 0.754   & 0.650   & 0.574   & 0.515   & 0.468   & 0.429   & 0.397   & 0.370   & 0.346
\end{tabular}
\end{center}
\rule{\textwidth}{0.90pt}
\end{table*}
%%%%%%%%%%%%%%%%%%%


There are two problems with this metric. The first involves the relationship between varying forms of 
partition differences and the resulting value of  $VI(\calp, \calp^\prime)$. To clarify the issue, suppose 
$n$ is an even integer, and define partitions on $\{ 1,\ldots, 2n \}$: $\calp_1 = ( \{1,\ldots, n\}, \{n+1, 
\ldots, 2n\}  )$,  $\calp_2 = ( \{1, \ldots, n/2\}, \{n/2+1, \ldots,  n\}, \{ n+1, \ldots, 2n\} )$, $\calp_3 
= ( \{1, \ldots, n-1\}, \{n , \ldots, 2n\} )$. $\calp_4 = ( \{1, \ldots, n-1, n+1\}, \{n , n+2, \ldots, 2n
\} )$. To fix ideas, this may correspond to a case in which a true partition $\calp_1$ consists of two 
sibling groups of size $n$,  $\calp_2$ is an estimate which incorrectly splits the first group into equal 
halves, $\calp_3$ is an estimate which incorrectly moves one individual from the first to the second group, 
and $\calp_4$ is an estimate which incorrectly exchanges two individuals between groups. These partitions 
are represented schematically in Table \ref{pex.table}.  Values of $VI(\calp_1, \calp_j)$ for $n = 4, 6, 
\ldots, 24$ are given in Table \ref{vi.table}. First note that the distributions induced by $\calp_1, 
\calp_2$ do not vary with $n$, so neither does $VI(\calp_1, \calp_2)$. On the other hand, $VI(\calp_1, 
\calp_3)$ decreases with $n$, so that $VI(\calp_1, \calp_2) <  VI(\calp_1, \calp_3)$ for $n \leq 8$. 
However, it seems unreasonable to score  $\calp_2$ as a more accurate estimate than $\calp_3$, especially 
for sibling groups of size $n = 8$ (the PD would assign errors of 4 and 1 respectively).  The situation is 
more severe for the case of $\calp_4$, for which $VI(\calp_1, \calp_2) <  VI(\calp_1, \calp_4)$ for all $n 
\leq 22$ (for $n = 22$ the respective PD errors would be 11, 1 and 2 when comparing $\calp_1$ to $\calp_2$, 
$\calp_3$ and $\calp_4$). 


%%%%%%%%%%%%%%%%%%%%%%%%%%%
\begin{table*}
\caption{Comparison of PRT and RGOA  using simulated data from 6 pedigree models (Table 5 \citep{Chou2012}). 
The table includes the number of replications (N Repl.), the time taken for the study (total of all 
replications) and the mean accuracy based on partition distance. For PRT, total number of correct precdigree 
estimates is also given (partition distance equals 0). For details, see Appendix 3.}\label{rgoa.table}
\rule{\textwidth}{0.90pt}
\begin{center}
\begin{tabular}{c|cccc|ccc}
\multicolumn{2}{c}{}&\multicolumn{3}{c}{PRT}&\multicolumn{3}{c}{RGOA} \\ \hline	
	&N 	&	&N	&Time	&&N &Time \\ 
Model	&Correct&Accuracy	&Repl.	&(sec)	&Accuracy	&Repl.&(sec) \\ \hline
Rand-j10-o40-l2-a10	&4	&93.45	&100	&1563.7	&91.00	&24	&$> 72,000$ \\
Rand-j10-o50-l2-a10	&6	&94.32	&100	&1981.0	&91.60	&13	&$> 72,000$ \\
Rand-j10-o40-l3-a10	&58	&99.63	&100	&1617.5	&100.00	&7	&$> 72,000$ \\
Rand-j10-o50-l3-a10	&57	&99.58	&100	&2120.1	&99.80	&7	&$> 72,000$ \\
Rand-j10-o40-l4-a10	&94	&99.98	&100	&1589.6	&100.00	&3	&$> 72,000$ \\
Rand-j10-o50-l4-a10	&95	&99.97	&100	&2134.2	&100.00	&3	&$> 72,000$
\end{tabular}
\end{center}
\rule{\textwidth}{0.90pt}
\end{table*}
%%%%%%%%%%%%%%%%%%%%%%%%%%%%%%%%%%%


A second problem involves the normalization. In \citet{BrownDexter2012} it is noted that $VI(\calp, \calp^
\prime)$ is bounded above sharply by $\log n$, so this is used as a normalizing constant, forcing the metric 
into the unit interval. The problem with this can be seen with the example  $VI(\calp_1, \calp_2)$. Since 
this value does not depend on $n$, the quantity $[\log n]^{-1} VI(\calp_1, \calp_2)$ will approach 0 as $n$ 
increases, even though there is no sense in which the estimate can be described as becoming more accurate. 
In contrast, the quantities $VI(\calp_1, \calp_3)$ and  $VI(\calp_1, \calp_4)$ do approach 0 without any 
normalization, sensibly, since the number of incorrectly assigned individuals (1 and 2, respectively) are an 
increasingly small proportion of $n$. This renders accuracy reports in Table 1 of \citet{BrownDexter2012} 
difficult to intepret across varying problem sizes. Because $VI(\calp, \calp^\prime)$ is bounded by joint 
entropy $H(\calp, \calp^\prime)$ a more natural normalization scheme would be $[H(\calp, \calp^\prime)]^{-1} 
VI(\calp, \calp^\prime)$, as proposed in \citet{kraskovEtAl2005}. This is preferable to $\log n$, which 
represents a worst case scenario that is remote from any estimation problem involving discernable sibling 
structure. 

 
We finally note that the choice of partition distance should anticipate the forms of error most likely to 
occur. Calculations in \citet{almudevar99} show that spurious additions to true sibling groups will be small 
in number. The splitting of true sibling groups is more pervasive, so any alternative partition distance 
should be sensitive to this.    

\section*{Appendix 3 - A comparison with the Randomized Greedy Optimization Algorithm (RGOA)} 

The authors of Kinalyzer have more recently proposed a modified formulation of the algorithm underlying 
KINALYZER, which introduces statistical as well as combinatorical optimization criterion (see discussion in 
\cite{alm&and11}). This in turn is modeled as a \emph{capacitated clustering problem} (CCP), a type of 
constrained clustering problem \cite{Chou2012}. This yields a mixed-integer nonlinear programming (MINLP) 
problem. The implementation is referred to by the authors as a  \emph{randomized greedy optimization 
algorithm} (RGOA). 

We offer a comparison of PRT to an evaluation of RGOA based on simulated data (Table 5, \cite{Chou2012}). 
The model parameters, based on the simulation algorithm of Section 5.2 of \cite{Chou2012}  are taken to be
\begin{eqnarray*}
j & = & \mbox{Number of sibling groups} \\
o & = & \mbox{Number of siblings per group} \\
a & = & \parbox[t]{.395\textwidth}{Number of alleles per locus (assumed uniformly distributed)}\\
l & = & \mbox{Number of loci (assumed unlinked)}. 
\end{eqnarray*}
Table \ref{rgoa.table} lists the models simulated in \citet{Chou2012}, using the nomenclature of Table 5. 
For example, `Rand-j10-o40-l2-a10' refers to the model defined by parameters $j = 10$, $o = 40$, $l = 2$ and 
$a = 10$. For each model PRT simulated 100 replications using the simulation utility \cite{alm&and11} (a 
standard Windows based PC was used). In each case the accuracy was assessed using the pedigree distance $PD$ 
(Accuracy = $(j\times o - PD)/(j \times o) \times 100\%$). Table \ref{rgoa.table} summarizes for each model 
the total number of correct estimates of the 100 replications ($PD = 0$), the mean accuracy, and the total 
time in seconds for the total simulation study. For example, the time taken to simulate 100 replications of 
model `Rand-j10-o50-l4-a10', calculate the estimates and evaluate $PD$ was 2134.2 seconds, so that a single 
MSG estimate averaged within 21.4 seconds. The accuracy attained by the RGOA was comparable, however the 
computation time far exceeded that attained by PRT. Our understanding of the study is that the algorithm was 
allowed to estimate sibling partitions of model replications within an allowed time of 72,000 seconds (= 20 
hours). In that time, for model `Rand-j10-o50-l4-a10' 3 estimates were completed, so that 5 hours is the 
most favorable report of the average computation time for an estimate. This exceeds by far the 21.4 seconds 
attained by PRT, with no apparent compromise in accuracy. Essentially the same conclusion may be reached for 
the remaining models. We conjecture that both algorithms attain close to the optimal accuracy given the 
inherent statistical error of the problem (see \cite{almudevar99} for methods with which to estimate this 
type of error).        

It must be noted that summaries of the performance of PRT given in Table 4 of \citet{Chou2012} (identified 
as `A\&F') are misleading, which presumably accounts for the failure in \citet{Chou2012} to accept PRT as a 
performance benchmark. The PRT algorithm was originally proposed in \citet{almudevar99}. In that work, a 
modification of the algorithm was described which permitted application to larger data sets. In particular, 
this modification enabled PRT to complete an estimate for the `Salmon' data cited in Table 4 of 
\citet{Chou2012}. The modified algorithm completed the estimate in approximately 11 hours using a Sun 
SPARCstation 4 (a relatively high performance computer for the year 1999). The most recent version of PRT 
completes the same estimate in 34.1 seconds \citep{alm&and11}. As reported in \citet{almudevar99}, this 
modification was needed to complete that estimate. We must conclude, therefore, that the authors of 
\citet{Chou2012} have not correctly implemented the algorithm as described in \citet{almudevar99}. 

\section*{Appendix 5 - PRT Parameter Settings} 

As discussed in Appendix 1, PRT currently supports tolerance for genotyping error. A parameter $K$ may be 
specified, giving the maximum number of genotype errors permitted per individual. Here, an error refers to 
an allele incompatible with Mendelian inheritance. A set of individuals is a FSG as long as it contains no 
more than $K$ errors in any individual.  

In order to determine the values of $K$ for this study the following procedure was used. For each allele/
loci/  error model PRT was run for 50 OneBig models using error parameter values $K =  1,2,3,4$. A 
\emph{minimax} criterion was then used for selecting $K$, that is, for each allele/loci model the value of 
$K$ minimizing the maximum median partition distance across the three error models was used. The value of $K
$ for the 20/20 model was changed from 3 to 4 to preserve monotonicity. 

For all models, the modified MSG-MG algorithm was used. For most models the MSG-MG algorithm used the 
default settings (see documentation): 
\begin{eqnarray*}
\mbox{Minimum MSG Size} &=& \mbox{number of individuals in sample} \\
\mbox{Min MSG Size Increment} &=&  10  \\
\mbox{Number of MSG Iterations} &=& 
 2 + \mbox{(Minimum MSG Size � 4) div 10}  
\end{eqnarray*}
with remaining options set to
\begin{eqnarray*}
\mbox{Maximum Partition List Size} & = & 5000 \\
\mbox{Max Sib Size for Full Search} & = & 10 \\
\mbox{Reduced Search Deletion} & = & 1 \\
\mbox{Number of Error Iterations} & = & 20.
\end{eqnarray*}
However, for the LottaLarge models with 20 or 25 loci, the following options were used
\begin{eqnarray*}
\mbox{Minimum MSG Size} & = & 25 \\
\mbox{Min MSG Size Increment} & = & 5 \\
\mbox{Number of MSG Iterations} & = & 4 \\
\mbox{Maximum Partition List Size} & = & 10000 \\
\mbox{Max Sib Size for Full Search} & = & 5 \\
\mbox{Reduced Search Deletion} & = & 1 \\
\mbox{Number of Error Iterations} & = & 1.
\end{eqnarray*}
These settings reduced the memory requirements for these more computationally challenging models. 

% latex table generated in R 3.0.1 by xtable 1.7-3 package
% Sat May 03 14:19:16 2014
\begin{table*}
\caption{Maximum number of genotype errors $K$ used for evaluation study of PRT, for allele/loci model. The 
value of $K$ was selected using the minimax selection procedure described in Appendix 5. The value of $K$ 
for the 20/20 model was changed from 3 to 4 to preserve monotonicity. The minimax median $PD$ is also 
listed.}\label{k.table}
\centering
\begin{tabular}{r|rrrrr|rrrrr} \hline
  & \multicolumn{5}{c|}{$K$} & \multicolumn{5}{c}{Minimax median $PD$} \\
 & \multicolumn{5}{l|}{$N$ loci =} & \multicolumn{5}{l}{$N$ loci =} \\
$N$ alleles  & 5 & 10 & 15 & 20 & 25 & 5 & 10 & 15 & 20 & 25 \\ 
  \hline
5 & 1 & 1 & 1 & 2 & 2 & 43.0 & 38.0 & 27.0 & 25.0 & 23.0 \\ 
  10 & 1 & 2 & 3 & 3 & 4 & 28.5 & 1.0 & 0.5 & 0.0 & 0.0 \\ 
  15 & 1 & 2 & 3 & 3 & 4 & 23.0 & 1.0 & 1.0 & 0.0 & 0.0 \\ 
  20 & 1 & 3 & 3 & 4 & 4 & 2.0 & 0.0 & 0.0 & 0.0 & 0.0 \\ 
  25 & 2 & 3 & 3 & 4 & 4 & 0.0 & 0.0 & 0.0 & 0.0 & 0.0 \\ 
   \hline
\end{tabular}
\end{table*}


\section*{Appendix 6 - AA's take} 

Given data of sufficient quality ($\geq 10$ alleles and $\geq 15$ loci) PRT can be generally recommended for 
all models except those dominated by small sibling groups. Interestingly, given it's emphasis on large 
sibling groups, PRT consistently performs as well, or better, than all other algorithms for the NoSibs, 
AllHalf and AllPatHalf models, across all error models. 

For larger sibling groups the situation varies (we again consider $\geq 10$ alleles and $\geq 15$ loci). For 
the OneBig and OneBig\_H models PRT appears to be consistently most accurate, although the accuracy of CO 
and FF is comparable. For the BigSGs and BigSGs\_H models CO offers the best accuracy consistently, although 
CP, PRT and FF remain viable alternatives. For these two models, memory requirements sometimes hampered the 
performance of PRT. For especially large problems careful tuning of memory parameters will be required. 

For the SmallSGs and  SmallSGs\_H model CO and CP was consistently most accurate.   



\bibliographystyle{men}
{\footnotesize
\bibliography{SibEvalBib.bib}}


\end{document}
